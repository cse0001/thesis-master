\begin{cabstract}
在现代软件开发中,使用第三方包成为了软件复用的重要手段。随着软件领域的不断演进,各类软件生态的通用做法是将越来越多的第三方包发布并维护在一个中央仓库中,以便开发者下载和安装。这种方式极大的提高了软件开发效率,但也给开发者维护第三方包兼容性带来了新的挑战。

目前,随着编程语言的飞速发展,许多语言都建立了自己的软件生态,与此同时,操作系统作为基础软件的代表,也演化出了以各种linux发行版为代表的操作系统生态。
两类生态都将可复用的软件代码打包并发布在公开的第三方包仓库中,并且它们一般都提供了服务于自身生态仓库的包管理工具,帮助用户管理第三方包并处理第三方包的依赖关系,
这些工具通常都经过精心设计来处理仓库内部的第三方包依赖关系,但是不考虑仓库间的第三方包依赖关系。
然而,由于用户往往需要使用来自不同仓库的软件包,可能会出现跨软件生态的兼容性问题(CC问题)。现有研究通常只关注单一软件仓库,无法检测CC问题。为了填补这一空白,本文以Python语言生态和Ubuntu系统生态为代表,将重点关注以下两个研究问题:(1)不同软件仓库中的第三方包发生兼容性问题的根本原因是什么?(2)如何预测、检测和修复CC问题?

基于以上问题和总结,本文深入分析CC问题,并进行了如下探索性工作:
\begin{itemize}
	\item 本文从Ubuntu系统软件仓库,Python软件仓库和Python解释器出发,研究不同仓库间Python软件包的兼容性问题。首先,本文系统性的探讨了Ubuntu包管理工具apt和Python包管理工具pip的依赖解析规则,然后深入分析了系统级Python解释器的包调用规则。基于上述分析,本研究总结了CC问题的触发模式和表现形式,并在Ubuntu生态软件仓库和Python软件仓库间,建立了一个包含1692个CC问题的数据集。
	\item 在上述分析的基础上,本研究设计了一款自动解决 CC 问题的工具 Hera。Hera 首先离线建立一个跨生态兼容性数据库、然后在线预测、检测和修复用户系统环境中的 CC 问题。实验结果显示,Hera在上述CC问题数据集上的检测准确率为 90.5\%,召回率为 93.7\%。我们还从GitHub和Stack Overflow收集并复现了26 个CC问题,Hera成功检测出了所有这些问题并提供了正确的修复建议。
\end{itemize}
\end{cabstract}
\ckeywords{Ubuntu生态; Python生态; 兼容性问题}

\begin{eabstract}
In modern software development, using third-party packages has become an important means of software reuse.
With the continuous evolution of the software field, the common practice across various software ecosystems is to publish and maintain an increasing number of third-party packages in a central repository for developers to download and install.
This approach significantly enhances software development efficiency, but it also introduces new challenges for developers in maintaining compatibility of third-party packages.

At present, with the rapid development of programming languages, many languages have established their own software ecosystems. At the same time, as a representative of foundational software, operating systems have also evolved into various Linux distributions, forming distinct OS ecosystems.
Both types of ecosystems package reusable software code and publish it in public third-party repositories, and they generally provide package management tools that serve their own ecosystem repositories to help users manage third-party packages and handle their dependencies.
These tools are usually designed with great care to handle intra-repository third-party package dependencies but do not consider inter-repository package dependencies.
However, since users often need to use packages from different repositories, cross-ecosystem compatibility issues (CC issues) may arise.
Existing research usually focuses on a single software repository and cannot detect CC issues.
To fill this gap, this paper takes the Python language ecosystem and the Ubuntu system ecosystem as representatives and focuses on the following two research questions: (1) What are the root causes of compatibility issues between third-party packages in different software repositories? (2) How to predict, detect, and resolve CC issues?

Based on the above questions and summary, this paper conducts an in-depth analysis of CC issues and undertakes the following exploratory work:
\begin{itemize}
	\item This paper investigates compatibility issues between Python packages across different repositories, starting from the Ubuntu system software repository, Python software repository, and Python interpreter.
	First, this paper systematically explores the dependency resolution rules of Ubuntu's package manager apt and Python's package manager pip, and then delves into the package invocation rules of the system-level Python interpreter.
	Based on the above analysis, this study summarizes the triggering patterns and manifestations of CC issues and establishes a dataset containing 1,692 CC issues between the Ubuntu ecosystem software repository and the Python software repository.
	\item Based on the above analysis, this study designed an automated tool named Hera to resolve CC issues.
	Hera first establishes an offline cross-ecosystem compatibility database, and then predicts, detects, and resolves CC issues in the user's system environment online.
	Experimental results show that Hera achieves a detection accuracy of 90.5\% and a recall rate of 93.7\% on the aforementioned CC issue dataset.
	We also collected and reproduced 26 CC issues from GitHub and Stack Overflow, and Hera successfully detected all these issues and provided correct repair suggestions.
\end{itemize}


\end{eabstract}
\ekeywords{Ubuntu Ecosystem; Python Ecosystem; Compatibility Issues}

