
\begin{resume}
   %评阅版论文隐去阶段性成果具体信息,保留此段文字:
	
  该论文作者在学期间取得的阶段性成果(学术论文等)已满足我校硕士学位评 阅相关要求。为避免阶段性成果信息对专家评价学位论文本身造成干扰,特将论文作者的阶段性成果信息隐去。
\ignore{
  \section*{发表的学术论文} % 发表的和录用的合在一起

  \begin{enumerate}[label={[\arabic*]},itemsep=0pt,parsep=0pt,labelindent=26pt,labelwidth=*,leftmargin=0pt,itemindent=*,align=left]
   %[label=\textbf{[\arabic*]},itemindent=*, align=left] %老版本缩进对齐
   
  %\addtolength{\itemsep}{-.36\baselineskip}%缩小条目之间的间距,下面类似
  \item \textbf{Yifan Xie}, Zhouyang Jia, Shanshan Li, Ying Wang, Jun Ma, Xiaoling Li, Haoran Liu, Ying Fu, Xiangke Liao. "How to Pet a Two-Headed Snake? Solving Cross-Repository Compatibility Issues with Hera", the 39th IEEE/ACM International Conference on Automated Software Engineering (ASE 2024), 27 October-1 November, 2024, Sacramento, California, United States. DOI: 10.1145/3691620.3695064. (CCF A类推荐会议)
  \end{enumerate}

  \section*{研究成果} % 有就写,没有就删除
  \begin{enumerate}[label={[\arabic*]},itemsep=0pt,parsep=0pt,labelindent=26pt,labelwidth=*,leftmargin=0pt,itemindent=*,align=left]
  %[label=\textbf{[\arabic*]},itemindent=*, align=left] %老版本缩进对齐
  %\addtolength{\itemsep}{-.36\baselineskip}%
  \item 李姗姗, 董威, 贾周阳, 马俊, 李小玲, 张元良, 王腾, 谢一帆, 黄响兵. 一种面向IO大小的数据库性能问题检测方法: 中国, CN116225965B. (中国专利公开号.)
  \item 李姗姗, 张元良, 李解, 王腾, 陈立前, 方寸谛, 谢一帆, 胡柳敏, 黄响兵. 一种面向配置缺陷的定向模糊测试方法: 中国, CN116841886B. (中国专利公开号.)
  \end{enumerate}
}
\end{resume}

