\chapter{绪论}

本章的主要内容与学校提供的Word模板中内容一致,图片与表格均采用原始设定大小,%
主要是为了说明格式的统一。%
但是,\LaTeX{}的一些禁则,专业排版的能力,对公式及文献的处理都是得天独厚的,%
我们不必刻意去追求与Word的完美匹配。而且你将会发现,用\LaTeX{}书写论文的美! %

\section{研究背景及意义}
兼容性问题是计算机软件工程领域的一个研究热点,它通过依赖冲突检测或程序分析等方式检测第三方库和第三方库、软件和第三方库之间的兼容性。第三方库 (又称第三方依赖或软件包) 是一种重要的可复用软件资源, 在一定条件下, 可以独立于其他第三方库进行安装和删除[1]。软件开发过程中调用第三方库是一种提高开发效率的普遍做法, 特别是随着现代信息技术的迅速发展, 软件规模不断增加, 软件开发对第三方库的依赖也在不断增加。已有研究工作表明一个项目直接依赖于多个不同的库[2]。此外, 由于第三方库之间存在依赖关系,一个项目往往会被隐蔽式的加入更多的依赖关系,导致依赖更多的库[2-5]。

第三方库的使用大大提升了软件开发效率,但也给软件开发带来了潜在的风险。当加载的第三方库无法覆盖所在项目或其他第三方库的必需特性(例如,方法) 时,就会发生兼容性问题。目前对兼容性问题的解决方式多为开发人员借助经验知识去处理,会消耗大量的时间和人工成本。运行过程中的报错信息在多数情况下不足以支撑开发人员找到问题根源,并且由于第三方库之间存在复杂的调用关系,更改某一第三方库同时可能引起其他第三方库的不适用,这些问题增加了检测和解决兼容性问题的难度,同时兼容性问题会伴随如下特点:难以重现问题出现的场景、编译器提示的问题报告中描述信息不够清晰、第三方库代码缺失、第三方库代码复杂难以调试等。这些问题在开发过程中给开发人员带来了巨大挑战。

随着现代软件开发的发展,各类软件生态的通用做法是将越来越多第三方库打包(第三方包)并维护在一个中央仓库,以便开发者下载和安装。为了帮助开发者处理第三方包间复杂的依赖关系,避免兼容性问题,软件生态提供了包管理器进行第三方包的下载,删除,依赖解析等。例如,在编程语言社区中,pip作为Python语言的包管理器,管理着接近五十万个包[6],npm作为JavaScript语言的包管理器,管理着超过三百万包。同样,在操作系统发行版社区中,apt作为Ubuntu系统的包管理器,管理着超过八万deb包,dnf作为Fedora的包管理工具,管理着超过七万rpm包。

包管理器设计的主要目标是在其软件生态内部进行正确的依赖解析,即避免引入的第三方包间发生不兼容问题。为了促进这一目标的达成,大量研究致力于提升对于第三方包的依赖管理能力。对于各种编程语言软件仓库,一系列工作[7-10]针对增强python包管理器的依赖解析和兼容性检测,部分研究[11-13]针对JavaScript软件仓库的依赖管理。还有部分研究[1,14-16]针对操作系统发行版社区软件仓库中的第三方包兼容性问题。然而,系统内单个软件仓库内部的包兼容性无法保证系统整体的包兼容性。开发者在实际软件开发中,往往需要结合多个软件仓库的资源,这就导致了一个研究空白:如何有效地分析和检测多个软件生态共同作用下的兼容性问题。而现有包管理器和研究工作大多致力于提高特定软件仓库的包管理和兼容性检测能力,这些努力通常局限于单一的软件生态,无法有效检测和修复多个软件生态共同影响下的包兼容性问题。

因此,针对这一研究空白,探究一种有效分析和检测跨软件生态的兼容性问题的方法是十分有必要的。这不仅对于提高软件系统的稳定性和性能有着直接影响,还有助于减轻开发人员的工作负担,优化开发流程,降低因兼容性问题带来的风险和成本。


\section{相关研究现状}
\subsection{依赖冲突检测方法}
现有大量工作致力于单个软件生态中的依赖冲突检测,并尝试对依赖冲突问题生成崩溃信息的堆栈跟踪。EVOSUITE[17,18]是一个面向Java语言的开源工具,采用遗传算法为给定的目标类生成测试用例集,通过自定义的适应度函数以及设计的交叉运算和变异函数, 加速搜索过程, 指导第三方库被调用方法的测试用例的生成, 提高对第三方库方法间冲突检测的有效性和效率使用测试的方式提高对第三方库方法间冲突检测的有效性和效率。Wang等人[19]提出的RIDDLE基于静态分析,构造控制流依赖关系,对依赖关系生成变体进行测试,是EVOSUITE的改进。

Patra等人[12]提出的ConflictJS针对的是JavaScript语言中,由于第三方库共享相同的全局命名空间导致的依赖冲突问题。这类问题的出现,是由于不同库对同一命名空间进行写操作。ConflictJS动态地分析单个第三方库写入全局命名空间的位置,基于库的全局写入位置进行匹配,进一步比较库中方法的行为,当且仅当该方法发现有不同行为时,报告依赖冲突问题。Wang等人[10]提出的Watchman是一个持续监测 PyPI 生态系统依赖冲突的工具。通过收集每个第三方包的metadata (元数据) 并进行广度优先搜索,为项目依赖的第三方包构建完整的依赖调用关系图。依赖关系图的实际呈现形式为有向无环图,图中的节点代表第三方库,节点间的有向边代表着两个包之间的依赖关系,有向边箭头指向的节点代表被依赖的第三方包。对依赖关系图中有多个有向边所指向的节点,即为被多个包同时依赖的第三方包; 分析其传入的边的集合,对边之间约束的版本关系进行匹配分析,若传入边之间约束的交集为空,证明存在依赖冲突问题。

Huang等人[20]通过分析项目代码中各个模块的第三方包版本,检测项目中的依赖冲突问题。遍历所有模块所依赖的包版本,对每一个包进行版本匹配,分析这些包的版本是否一致,如果不是,那么检测到了包版本不一致的问题,并报告其所影响的模块。根据匹配结果,按照最少改动的原则,进行版本的推荐Mancinelli等人[21,22]提出的EDOS是基于形式化方法,对第三方包中的依赖冲突关系进行解析,将第三方包的可安装性问题转化为约束求解问题,并设计为自动化的检测工具。从操作系统发行版的角度,降低包之间冲突出现的频率,提高了发布的第三方包的稳定性。
\subsection{兼容性问题检测方法调研}
对于项目或第三方包之间的兼容性问题,现有方法总体上基于匹配的思想,对比不同版本第三方包的运行结果,或构建项目方法的依赖调用图或抽象语法树,进行调用关系的对比匹配。Foo等人[23]提出的JTEXPERT是一种自动化软件测试数据生成方法,它实现了单元类测试的高代码覆盖率。其核心思想是基于遗传算法,通过类实例生成器和种子策略来优化搜索,并通过静态分析方法从搜索空间中删除不相关的输入变量加速搜索过程,其生成的测试数据可用于检测不同版本的第三方包和调用关系不一致的行为。

一种常见的兼容性问题表现为,开发人员期望加载的版本和实际加载的版本中,使用的方法具有相同的签名,但是两者程序运行行为不一致。Wang等人[24]提出的SENSOR使用遗传算法来检测API之间的兼容性。SENSOR首先从源代码中提取每个对象的构造函数和API调用的上下文,并利用它们来构造类实例池和API参数池;基于GUMTREE[25]迭代检测得到的代码差异,在细粒度级别上识别异构的冲突API对。SENSOR认为异构的冲突API对可能会导致兼容性问题,SENSOR方法将类实例调用的种子策略与EVOSUITE[17]结合起来,生成测试用例集来触发相关的库API,并检查它们在不同版本中的行为是否一致。

PYCOMPAT[26]运用静态分析的方法,检测由Python第三方库中API更改引起的兼容性问题。具体来说,是为了检测由API重命名和参数重命名引起的问题,其检测分为两个阶段:第1阶段抽取第三方库中的API更改信息,当API的更改信息无法进行自动化抽取时,采用人工抽取;对抽取出的信息,构建知识库,第2阶段将 API 知识库作为输入,并对给定的Python源文件执行静态分析,构建抽象语法树(AST),遍历AST以获得框架中定义的API的调用,进而通过定义的匹配规则来检查对API的调用,是否使用了改进的API,定位兼容性问题。
Foo等人[27]运用静态分析方法来检查第三方库在升级后,是否引入了不兼容库中API的方法。其首先构建程序的库调用图,利用Myers算法提取升级前后库中API的差,于版本的更改,可能直接跳跃多个版本。算法计算每个相近版本间API的差异,进而得到最终的差异结果,并结合依赖调用图来判断是否发生了不兼容的更改。通过解析配置文件,对改动最少的版本进行推荐。
\subsection{第三方包演化研究}
兼容性问题主要体现于API之间的不兼容,调研第三方包演化过程中API兼容兼容性问题的相关研究工作,可以为检测兼容性问题提供参考性意见。Meng 等人[28]提出的HiMa基于历史的匹配方法来识别和理解Java框架的API演变。Hora 等人[29]提出的APIwave是一个基于匹配规则,识别跟踪API流行度和迁移的大型工具,与HiMa的不同点在于,APIwave的使用范围更广,可以跟踪API的流行程度。APIwave与HiMa对第三方库的对比,可以帮助开发人员快速了解两个版本之间的差异,进而修改自己的代码。当开发人员面临第三方包之间的兼容性问题时,可以使用此工具快速发现新版本中产生变更的API,来修改本地项目代码,减少兼容性问题的发生。

Ponomarenko等人[30]提出了一种在二进制级别的、可适用于多种语言的、自动检测第三方库的向后兼容性问题的新方法。此方法除了分析组件二进制文件中的符号外,还可以通过比较从组件头文件中获得的函数签名和类型定义,来验证向后兼容性问题。Jia等人[34]同样提出了二进制级别的第三方包不兼容更改检测工具,可以检测第三方包演化过程前向和后向不兼容问题。这种二进制级别的兼容性检测相比于API级别的检测更细粒度也更准确。

NoRegrets[31]是一个回归测试工具,可以用来确定Javascript第三方包更新后是否影响了有关API的使用,并对更新前后API的兼容性进行对比。NoRegrets+[32]通过分析Javascript第三方包被其他第三方包复用的情况,自动判断第三方包的更改是否影响公共API的使用,并自动生成测试用例,来查找库中的破坏性更改。Lamothe等人[33]基于文档和历史代码更改信息,对Android中API的迁移升级方法进行检测。

综上,现有相关工作主要聚焦于单个软件生态中的兼容性问题的分析、检测和解决,对于跨软件生态的兼容性问题尚未由成熟的解决方案。因此,探究一种有效分析和检测跨软件生态的兼容性问题的方法,具有相当的研究意义和实践意义。

\ignore{
\begin{table}[htp]
	\centering
	\begin{minipage}[t]{0.8\linewidth} % 如果想在表格中使用脚注,minipage是个不错的办法
		\caption[表 1.1 名称]{}
		\begin{tabular*}{\textwidth}{lp{10cm}}
			\toprule[1.5pt]
			{\hei 列1} & {\hei 列2} \\
			\midrule[1pt]
			&  \\
			& \\
			& \\
			& \\
			& \\
			& \\
			\bottomrule[1.5pt]
		\end{tabular*}
	\end{minipage}
\end{table}
}

\section{研究内容和贡献}
\section{论文结构}

